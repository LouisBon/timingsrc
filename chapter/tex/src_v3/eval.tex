The evaluation is concerned with feasibility of the motion model, as well as
it two primary objectives; Web availability and simplicity for developers.

Sect.~\ref{sec:motionsync} discussed motion synchronization, and reported
synchronizations errors in the 0-5 millisecond range. Typically we observe 0-1
millisecond errors for desktop browsers, compared to a system clock
synchronized by NTS. These results are achieved using the \emph{InMotion}
service provided by Motion Corporation~\cite{mcorp}. This service has been running
continuously for years, supporting a wide range of technical demonstrations,
at any time, at any place, and across a wide range of devices. As such, the
value of a production grade service is also confirmed.

Sect.~\ref{sec:compsync} discussed synchronization of HTML5 media
elements, and reported how the MediaSync library produces synchronization
errors of about 7 milliseconds for both audio and video, as documented in
technical reports~\cite{syncreport1,syncreport2}. These results have been consistently
confirmed by day to day usage over several years. The user experience of
multi-device video synchronization is very good, to the point that errors are
hardly visible. The exception might be content with particular geometries and
rapid movements, as demonstrated by this video~\cite{carneval}. Synchronization
has also been maintained for hours and days at end, without any visible
errors. In addition, loading speeds are acceptable. Even though the MediaSync
library requires about 3 seconds to reach echoless, the experience is
perceived as acceptable much before this. A variety of video demonstrations
have been published at the Multi-device Timing Community Group
Website~\cite{mtcg}.

Interestingly, the results for motion synchronization and HTML5 media
synchronization are well aligned with current limitations of the Web platform.
For instance, the precision of timed operation in JavaScript is about 1
millisecond, and a 60Hz screen refresh rate corresponds to 16 milliseconds.
Furthermore, these results also match limitations in human sensitivity to
synchronization errors. Identical audio signals skewed by less than 7
milliseconds will likely be interpreted as natural echo by the brain, and
collapsed into one signal (with directional information)~\cite{syncreport2}.

Finally, programming synchronized media experiences in the motion model is
both easy and rewarding. In our experience, motions and sequencers are
effective thinking tools as well as programming tools. A globally synchronized
video experience essentially requires three code statements.

With this, we argue that the feasibility of the motion model is confirmed. Web
availability for media synchronization is demonstrated by the \emph{InMotion} hosting
service for online motions. It is also clear that synchronization errors in
online synchronization are currently dominated by errors in synchronization in
HTML5 media elements. Future standardization efforts and optimizations would
likely yield significant improvements.
