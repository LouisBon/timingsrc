This section lists abbreviations and central terms used in this chapter.

\subsection{Abbreviations}

\begin{itemize}
\item{\textbf{URL:} universal resource locator}
\item{\textbf{HTML5:} hypertext markup language, version 5}
\item{\textbf{W3C:} World Wide Web Consortium}
\item{\textbf{HTTP:} hypertext transfer protocol}
\item{\textbf{AV:} audio and/or video}
\item{\textbf{DOM:} document object model}
\item{\textbf{UI:} user interface}
\item{\textbf{API:} application programmer interface}
\item{\textbf{IP:} Internet protocol}
\item{\textbf{UDP:} user datagram protocol}
\item{\textbf{TCP:} transmission control protocol}
\item{\textbf{NTP:} network time protocol}
\item{\textbf{PTP:} precision time protocol}
\item{\textbf{RTT:} network round-trip time}
\item{\textbf{WAN:} wide area network}
\item{\textbf{EDGE:} Enhanced Data rates for GSM Evolution, typically lowest level of Internet connectivity for smart phones.}
\item{\textbf{JSON:} JavaScript object notation}
\item{\textbf{GPS:} global positioning system}
\item{\textbf{UTC:} coordinated universial time}
\item{\textbf{XML:} extendible markup language}
\item{\textbf{WiFi:} wireless local area networking}
\item{\textbf{CSS:}} cascading style sheets
\end{itemize}


\subsection{Central terms}

\runinhead{Timeline:} 

logical axis for media presentation. Values on the timeline are usually
associated with a unit, e.g. seconds, milliseconds, frame count or slide
number. Timelines may be infinite, or bounded by a range (i.e minimum and
maximum values).

\runinhead {Clock:} 

a point moving predictably along a timeline, at a fixed, positive rate.
Hardware clocks ultimately depend on a crystal oscillator. System clocks
typically count seconds or milliseconds from \emph{epoch} (i.e. 1. Jan 1970
UTC), and may be corrected by clock synchronization protocols (e.g.
NTP~\cite{ntp}, PTP~\cite{ptp}). From the perspective of application
developers, the value of a clock may be read, but not altered.

\runinhead {Motion:} 

a unifying concept for \emph{media playback} and \emph{media control}. Motion
represents a point moving predictably along a timeline, with added support for
flexibility in movement and interactive control. Motions support discrete
jumps on the timeline, as well as a variety of continuous movements expressed
through velocity and acceleration. Not moving (i.e. paused) is considered a
special case of movement. Motion is a generalization over classical concepts
in multimedia, such as \emph{clocks}, \emph{media clocks}, \emph{timers},
\emph{playback controls}, \emph{progress}, etc. Motions are implemented by an
\emph{internal clock} (seconds) and a \emph{vector} describing current movement (position,
velocity, acceleration), timestamped relative to the internal clock.
Application developers may update the movement vector of a motion at any time.

\runinhead {Timed data:} 

data whose temporal validity is defined in reference to a timeline. For
instance, the temporal validity of subtitles are typically expressed in terms
of points or intervals on a media \emph{timeline}. Similarly, the temporal validity
of video frames essentially maps to frame-length intervals. Timed scripts are
a special case of timed data where data represents functions, operations or
commands to be executed.


\runinhead {Continuous media:} 

Typically audio or video data. More formally; subset of \emph{timed data}
where media objects cover the timeline without gaps.

\runinhead {Discrete media:} 

E.g. subtitles or sensor data. Not \emph{continuous media}.
Media objects may be distributed scarcely, non-uniformly on the
timeline, or overlap.

\runinhead {User Interface (UI):} 

refers to the interface between a computer process and an end-user. Computer
processes might communicate with end-users visually, by audio, by vibration,
or other forms or cognitive stimulation. End-users may also provide feedback
to computer processes through the same interface, for instance by touching,
typing, clicking or talking. On the Web, UI is often represented through DOM
elements and/or JavaScript API’s.

\runinhead {Media Component:} 

essentially a player for some kind of timed data. Media components are based
on two basic types of resources; timed data and motion. The timeline of timed
data must be mapped to the timeline of motion. This way, motion defines the
temporal validity of timed data. At all times, the media component works to
produce correct media output in the UI, given the current state of timed data
and motion. A media component may be anything from a simple text animation in
a div element, to a highly sophisticated media framework.

\runinhead {{\textless}div{\textgreater}} 

The div tag is a frequently used tag in HTML. It defines a section in an
HTML document and is used to group block-elements to format them with CSS.

\runinhead {User Agent:} 

any software that retrieves, renders and facilitates end user interaction with
Web content, or whose user interface is implemented using Web technologies.

\runinhead {JavaScript:} 

(ECMAScript) programming language of the Web. 

\runinhead {Browsing context:} 

JavaScript runtime assosciated with Web document. Browser windows, tabs or
\emph{iframes} each have their own browsing context.

\runinhead {Iframe:} 

Web document nested within a Web document, with its own browsing context.
