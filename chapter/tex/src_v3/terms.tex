\begin{itemize}
\item{\textbf{epoch:} clock counting seconds (or milliseconds) since Jan 1. 1970}
\item{\textbf{timeline:} logical axis for media playback}
\item{\textbf{media clock:} clock representing media playback state, e.g. media offset}
\item{\textbf{media controls:} operations that alter a media clock, e.g. play/pause}
\item{\textbf{timed data:} data tied to points or intervals on the timeline}
\item{\textbf{continuous media:} ordered sequence of media objects, e.g. audio or video frames}
\item{\textbf{discrete media:} not continuous media, e.g media objects which overlap on the timeline, or are distributed scarcely or non-uniformly}
\item{\textbf{JavaScript:} programming language of the Web}
\item{\textbf{browser context:} JavaScript runtime hosted by a Web browser}
\item{\textbf{user agent:} Any software that retrieves, renders and facilitates end user interaction with Web content, or whose user interface is implemented using Web technologies.}
\item{\textbf{iframe:} Webpage within a Webpage, with its own browser context}
\item{\textbf{media component:} anything from a simple $<div/>$ element to a highly sophisticated media player or framework. Media components have \emph{media data} and \emph{media clock}. Media data
is organized according to a timeline, and media clock represents playback along
this timeline. Through its \emph{user interface (UI)}, media components
express \emph{media output} (e.g. pixels on screen, audio, vibration) and
receive \emph{media input} (e.g. key-presses, touch or mouse-events, camera, mic).}
\end{itemize}