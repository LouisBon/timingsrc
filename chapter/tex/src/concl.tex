In this chapter we presented the motion model, a general approach to media
synchronization on the Web. In the motion model, media clock and media
controls are merged into a single concept, motion, and made available as an
online resource. This ensures global synchronization and Web availability.
Media synchronization is simply a consequence of connecting multiple media
components to the same online motion. Furthermore, online motions are
represented in Web browsers by local motion objects. This way, media
synchronization is reduced to a local challenge, thereby shielding Web
developers from the complexities of distributed motion synchronization.
Existing media frameworks may be integrated into the motion model by
JavaScript wrapper code, yet internal support would likely be better.
Sequencing tools simplify the construction of custom media components,
effectively reducing the challenge of media synchronization to the challenge
of regular Web development.

The feasibility of the motion model is confirmed by evaluating synchronization
of HTML5 media elements connected to shared online motion. By this approach,
global, echoless audio and video synchronization is achievable across
different Web browsers, architectures and devices. The simplicity and broad
utility of the approach has also been confirmed through countless technical
demonstrations over the last few years, touching a wide range of use-cases in
digital multimedia media.

Importantly though, while quality of synchronization is key to determine
feasibility, it may not be the most relevant metric for the value of the
model. Likely, the extreme flexibility of the model is even more valuable. The
motion model allows complex multimedia experiences to be built by
synchronizing large numbers of dedicated, reusable media components,
specialized for a specific objective, data type and/or delivery method, across
devices globally, or within a single Web page. There is also great flexibility
in how media components may be added or removed dynamically, or perhaps
switched to a different motion. Furthermore, as any IP-connected device may
connect to an online motion, media experiences may cross platform boundaries
without even requiring any specific integration, perhaps making use of
dedicated visualization software or platform-specific content sources, while
remaining synchronized with Web-based media components.

Finally, modern multimedia products are facing increasingly tough
requirements. There are new data sources, new transfer protocols, new content
types and new visualization technologies. In addition, user demands are ever
rising; products must be immersive, personalized, interactive, exploit
secondary devices, and more. If media systems are not designed for
flexibility, extending them may be costly and time-consuming, eventually
leading to unmanageable complexity. In this context, the motion model shines.
