Dictionary definitions of \emph{media synchronization} typically refer to
presentation of multiple instances of media at the same time. A related term
is \emph{media orchestration}, possibly emphasizing more the importance of
media control and planned scheduling of media playback. Similarly, we have
used the term \emph{media timing} to highlight the universal utility of clocks
in digital media, for capture, control, synchronization, timeshifting,
scheduling and playback. In this chapter we use the term \emph{media
synchronization} in a broad sense, as a synonym to \emph{media orchestration}
and \emph{media timing}. We also limit the definition in two regards:

\begin{itemize}
\item{Media synchronization on the Web is clock-based. The latencies and heterogeneity of the Web environment requires a clock-based approach for acceptable synchronization quality.}
\item{Media synchronization involves one media instance and a clock. The term relative synchronization is reserved for comparisons between media instances.}
\end{itemize}

\subsection{Challenges}


\begin{table}
\centering
\caption{Common challenges for media synchronization on the Web.}
\label{tab:challenges}
\setlength{\tabcolsep}{10pt}
\begin{tabular}{cc}
\hline\noalign{\smallskip}
Synchronization challenges & Use-cases \\
\noalign{\smallskip}\svhline\noalign{\smallskip}
across media sources & multi-angle video, ad-insertion \\
across media types & video, WebAudio, animated map \\
across iframes & video, timed ad-banner \\
across tabs, browsers, devices & split content, interaction \\
across platforms & Web, native, broadcast \\
across people and groups & collaboration, social \\
across Internet & global media experiences \\
\noalign{\smallskip}\hline\noalign{\smallskip}
\end{tabular}
\end{table} 



Media synchronization has a wide range of use-cases on the Web, as illustrated
by Table~\ref{tab:challenges}. Well known use-cases for synchronization within
a single Web page include multi-angle video, accessibility features for video,
ad-insertion, as well as media experiences spanning different media types,
media frameworks, or iframe boundaries. Synchronization across Web pages allow
Web pages to present alternative views into a single experience; dividing or
duplicating media experiences across devices. Popular use-cases in the home
environment involve collaborative viewing, multi-speaker audio, or big screen
video synchronized with related content on handheld devices. Use-cases towards
the end of the list target global scenarios, such as distributed capture and
synchronized Web visualizations for a global audience.

\subsection{Approach}

The challenges posed by all these use-cases may be very different in terms of
complexity, requirements for precision, scale, infrastructure and more. Yet,
we argue that a single solution is needed. Implementing specific solutions for
specific use-cases is very expensive and time-consuming, and lays heavy
restrictions on reusability. Even worse, circumstances regarding
synchronization may change dynamically during a media session. For instance, a
smartphone involved in synchronization over the local network, will have to
switch approach for synchronization once the user leaves the house, or
switches from WiFi to the mobile network. Crucially though, by solving media
synchronization across Internet, all challenges listed above are solved by
implication. For instance, if video synchronization is possible across Web
pages on the Internet, then synchronizing two videos within the same Web page
is just a special case. It follows that the general solution to media
synchronization on the Web is distributed and global in nature. Locality may
be exploited for synchronization, yet only as optimization.
